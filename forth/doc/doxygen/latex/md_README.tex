

Author\-:


\begin{DoxyItemize}
\item Richard James Howe.
\end{DoxyItemize}

Copyright\-:


\begin{DoxyItemize}
\item Copyright 2013 Richard James Howe.
\end{DoxyItemize}

License\-:


\begin{DoxyItemize}
\item L\-G\-P\-L
\end{DoxyItemize}

Email(s)\-:


\begin{DoxyItemize}
\item \href{mailto:howe.r.j.89@googlemail.com}{\tt howe.\-r.\-j.\-89@googlemail.\-com}
\end{DoxyItemize}

\section*{Introduction}

This is a portable F\-O\-R\-T\-H implementation called \char`\"{}\-Howe Forth\char`\"{}, it is designed to be customizable to your needs, modular and fairly fast. Eventually it is intended that this will be ported to an embedded system (Most likely A\-R\-M based), for now it will stay on the desktop computer.

\subsection*{Directory structure}

Projects directory structure\-:

\subsubsection*{bin/}

This contains the binary that {\itshape should} have built without any problems what so ever, it also contains a system link into the \char`\"{}fth/\char`\"{} directory.

To this forth once it has been built change directories to \char`\"{}bin/\char`\"{} and type\-:

./forth

It will look for a file called \char`\"{}forth.\-4th\char`\"{} in the same directory which is the first thing it will execute.

\subsubsection*{doc/}

The manual(s) written in markdown are in this directory. They can be compiled to H\-T\-M\-L by typing \char`\"{}make html\char`\"{} which will create the documents in the same directory, \char`\"{}doc/\char`\"{}.

\subsubsection*{fth/}

The Forth source is contained in here! This is where most of the functionality is actually defined.

\subsubsection*{lib/}

\char`\"{}lib/\char`\"{} contains the Howe Forth library which \char`\"{}main.\-c\char`\"{} is simply a wrapper around, it defines the Forth virtual machine and implements the core primitives.

\subsection*{Requirements}


\begin{DoxyItemize}
\item G\-C\-C
\end{DoxyItemize}

Used to compile the C program.


\begin{DoxyItemize}
\item G\-N\-U Make
\end{DoxyItemize}

Used for the build system.

There are some optional dependencies that are more to do with debugging the system then anything else and are probably not relevant.

\subsection*{Optional requirements}

These are extra tools that can be used in conjunction with the project. In the make file there are commands that you can run that will act as wrappers around these tools.

\subsubsection*{markdown}

Typing {\bfseries make html} will create html files of all $\ast$.md in the directory.

\subsubsection*{gcov}

{\bfseries make gcov} will compile the forth interpreter and run it with the initial configuration file. \char`\"{}gcov\char`\"{} is then run showing what lines are executed most frequently.

\subsubsection*{valgrind}

{\bfseries make valgrind}

This compiles the program and runs it in valgrind, outputting everything to a file called {\itshape valgrind.\-log}. The program halts after the initial configuration file is given as with {\bfseries make gcov}.

\subsubsection*{G\-N\-U indent}

This will format the files in standard way, if you do not like the way my code looks (formatting wise!) you can change the command passed to indent in the make file and rerun {\bfseries make pretty}. This command will also clean up the directory structure like {\bfseries make clean} does as well as run \char`\"{}wc\char`\"{} over the sources files.

\subsection*{Notes}

To compile type {\bfseries make} and then to run type $\ast$$\ast$./forth$\ast$$\ast$. The makefile has more options in it which can be displayed with the command {\bfseries make help}.

The documentation is provided in three files\-: {\itshape \hyperlink{MANUAL_8md}{M\-A\-N\-U\-A\-L.\-md}}, {\itshape \hyperlink{README_8md}{R\-E\-A\-D\-M\-E.\-md}} (this file) and {\itshape \hyperlink{TODO_8md}{T\-O\-D\-O.\-md}}. You probably want to start off with the file {\itshape \hyperlink{MANUAL_8md}{M\-A\-N\-U\-A\-L.\-md}} which naturally contains the manual. {\itshape \hyperlink{FORTH_8md}{F\-O\-R\-T\-H.\-md}} contains a tutorial on my dialect of Forth in more details than {\itshape \hyperlink{MANUAL_8md}{M\-A\-N\-U\-A\-L.\-md}}. They might not be up to date however!

{\itshape \hyperlink{TODO_8md}{T\-O\-D\-O.\-md}}, {\itshape \hyperlink{MANUAL_8md}{M\-A\-N\-U\-A\-L.\-md}} and {\itshape \hyperlink{FORTH_8md}{F\-O\-R\-T\-H.\-md}} are all in the \char`\"{}doc/\char`\"{} directory.

Please contact me with any errors you encounter. 